\section{Numerical Analysis}

In this section look into the numerical analysis of the methods above, combined with the Krylov methods.
We will begin by presenting some already estabilished results for these methods.
Next we will attempt to extend these results in order to quantify the error produced by when using the Krylov methods for approximating the action of the exponential of a matrix to a vector.

\subsection{Formulation of the Problem}
Here we state the problem as formulated by \cite{Huang2022}.
We begin by considering a parabolic PDE with initial conditon of $u_0 \in H^2(\Omega) \cap H^1_0(\Omega)$ of the form:
\begin{align*}
    u_t &= D \Delta u + f(t,u) &x\in \Omega, 0 \leq t \leq T\\
    u(0,x) &= u_0(x) &x\in \Omega\\
    u(t,x) &= 0 &x\in \partial \Omega, 0 \leq t \leq T
\end{align*}
where $T$ is the final time, $\Omega$ is a rectangular domain in $\mathbb{R}^d$, $D\geq 0$, $u(t,x)$ is the unkown function and $f(t,u)$ is a nonlinear term.

The variational form of this problem is to find $u\in L^2(0,T;H^1_0(\Omega))$ and $u_t \in L^2(0,T: L^2(\Omega))$ such that
\begin{align*}
    (u_t, v) + a(u, v) &= (f(t,u),v) \forall v \in H^1_0(\Omega), 0\leq t \leq T\\
    u(0) &= u_0
\end{align*}
where $(\cdot,\cdot)$ is the $L^2$ inner product on $\Omega$.
The bilinear form $a(\cdot,\cdot)$ is given by
\begin{align*}
    a(w,v) &= \int_{\Omega} D \nabla w \cdot \nabla v dx, \forall w, v \in H^1_0(\Omega)
\end{align*}
and is clearly symetric.

We wil now begin to formulate the finite element space $V_h$ that we will approximate $H^1_0$ with.
As $\Omega \in \mathbb{R}^d$ is a rectangular domain we can assume that $\bar \Omega := \prod^d_{i=1}[a_i,b_i]$.
For every $i = 1,...,d$, we can define a uniform partition of $[a_i,b_i]$ with the subinterval size $h_i = \frac{b_i - a_i}{N_i}$,
to get the nodes $x^j_i = a_i + jh_i, j = 0,...,N_i$ as $a_i = x_i^0 < x_i^1 < ... < x_i^{N_i} = b_i$.
With this regular partition, we obtain a one-dimensional continuous piecewise linear finite element space for $[a_i,b_i]$ As
\begin{align*}
    V^i_{h_i}(a_i,b_i) :&= \{v\in C[a_i,b_i]: v|_{[x^{j-1}_i, x^j_i]}\in\mathbb{P}_1([x_i^{j-1},x_i^j]),1\leq j \leq N_i\} \cap H^1_0(a_i,b_i)\\
    &=\text{span}\{\theta^1_i(x_i), ... , \theta^{N_i-1}_i(x_i)\} 
\end{align*}
where $\theta^j_i(x_i)$ is the $j$-th nodal basis function of $V_h^i(a_i,b_i)$. 
Now we can write a finite element space for $\Omega$ as follows:
\begin{align*}
    V_h :&= V_{h_1}^1(a_1,b_1) \otimes ... \otimes V^d_{h_d}(a_d,b_d)\\
    &= \text{span}\{\theta_1^{i_1}\cdot \cdot \cdot \theta_d^{i_d}(x_d): 1 \leq i_1 \leq N_1-1,...,1\leq i_d \leq N_d-1\}
\end{align*}
We now have that $V_h\subset H^1_0(\Omega)$.
Let $h=\text{max}_{1\leq i\leq d}h_i$ be the mesh shize of the corresponding uniformly rectangular partition $\mathcal{T}_h$ that generates $V_h$.
All erorr analysis will assume $h\approx h_i$ for all $i$.

The finite element approximation for the above problem is then to find $u_h \in L^2(0,t:V_h)$ so that:
\begin{align*}
    (u_{h,t},v_h) + a(u_h,v_h) &=(f(t,u_h),v_h) \forall v_h \in V_h, 0\leq t \leq T\\
    u_h(0) &= P_hu_0
\end{align*}
where the $L^2$ orthogonal projection operator is given by $P_h:L^2(\Omega) \rightarrow V_h$.

(Need to expand, not quite complete)

Using the projection operator $P_h$ the problem can be rewritten to the equivalent system:
\begin{align*}
    u_{h,t} + L_hu_h &= P_hf(t,u_h)\\
    u_h(0) &= P_hu_0
\end{align*}

\subsection{Estabilished Results}
We now state the results given by Huang, et al\cite{Huang2022}.
Begining with some assumptions:

\begin{assumption}\label{assump:mildgrowth}
    The function $f(t,\zeta)$ grows mildly with respect to $\zeta$.
    That is there exists a number $p>0$ for $d=1,2$ or $p\in(0,2]$ for $d=3$ such that:
    \begin{align*}
        |\frac{\partial f}{\partial \zeta}| \lesssim 1 + |\zeta|^p. \quad \forall t,\zeta \in \mathbb{R}
    \end{align*}
\end{assumption}
\begin{assumption}\label{assump:smooth}
    The function $f(t, \zeta)$ is sufficiently smooth with respect to $t$.
    That is to say that for any given constant $K>0$ the following holds:
    \begin{align*}
        \sum_{|\alpha|\leq2}|D^{\alpha}f(t,\zeta)|\lesssim 1, \quad \forall t\in [0,T], \zeta \in[-K,K]
    \end{align*}
\end{assumption}
\begin{assumption}\label{assump:regularity}
    The exact solution u(t) satisfiies the following regularity conditions:
    \begin{align*}
        \text{sup}_{0\leq t \leq T}||u(t)||_{2,\Omega} \lesssim 1,\\
        \text{sup}_{0\leq t \leq T}||u_t(t)||_{L^\infty(\Omega)} \lesssim 1
    \end{align*}
    Where the hidden constant may depend on $T$.
\end{assumption}

We can now state the theorem that gives the error for the first order exponential integrator scheme.
\begin{theorem}
    Suppost the function $f$ satisfies Assumptions \ref{assump:mildgrowth}\ref{assump:smooth}\ref{assump:regularity}.
    There exists a constant $h_0 > 0$ such that for $h<h_0$, we have that the numerical solution given by:
    \begin{align*}
        u_h^{n+1} &= e^{-\tau L_h}u_h^n + \tau \phi_1(-\tau L_h)P_hf(t_n,u_h^n)
    \end{align*}
    satisfies the following:
    \begin{align*}
        ||u(t_n) - u_h^n||_1 &\lesssim \Delta \tau + h, \forall n =1,...,N_T
    \end{align*}
    with a hidden constant that is independatn of $\Delta \tau$ and $h$.
\end{theorem}

\subsection{Extension to Krylov Methods}

In this section, we will prove the main theorem that ties the error of the first order method to the Krylov methods.
We begin by stating some lemma that we will need to prove this result.
We ommit the proof of lemmas already proven by Huang\cite{Huang2022}.

\begin{lemma}
    Suppose that the function $f$ satisfies assumption \ref{assump:mildgrowth}, and the exact solution satisfies \ref{assump:regularity}.
    Then we have that $f$ is locally-Lipschitz continuous in a strip along the exact, i.e. for some $\epsilon > 0$ we have:
    \begin{align*}
        ||f(t,v) - f(t,w)||_0 &\lesssim ||v-w||_1
    \end{align*}
    for any $t\in [0,T]$ and $v,w \in V_h$ satisfying
    \begin{align*}
        \text{max}\{||v-u(t)||_1,||w-u(t)||_1\}\leq \epsilon
    \end{align*}
\end{lemma}

\begin{lemma}
    Consider a Krylov approximation given by $V^TAV = H$ with respect to a vector $v$.
    If $A$ is negative semi-definate then so is $H$.
\end{lemma}
\begin{proof}
    \begin{align*}
        x^T Hx &= x^T V^TAVx\\
        &= y^tAy \text{ where y = Vx}\\
        &\leq 0
    \end{align*}
\end{proof}

\begin{definition}
    We let $e_m^A$ denote the Krylov approximation of $e^A$ with respect to some vector $v$.
\end{definition}

\begin{lemma}
    We have that $||e_m^A|| \leq m$ when $A$ is negative semi-definate.
\end{lemma}
\begin{proof}
    \begin{align*}
        ||e_m^A|| &= ||e^{VHV^T}||\\
        &= ||Ve^HV^T||\\
        &\leq ||V||_2||e^H||||V^T||_2\\
        \intertext{we use the fact that $A$ and hence $H$ is negative semi-definate to imply that all of the eigen values $\lambda$ of $H$ are such that $\lambda \leq 0$
        and hence we have that $||e^H|| \leq ||e^0|| = 1$. From here we see that}
        &\leq ||V||_2||V^T||_2\\
        \intertext{we now use the fact that the $m$ collumns of $V$ are orthonormal to get}
        &= \sqrt{m} \sqrt{m}\\
        &= m
    \end{align*}
\end{proof}

\begin{definition}
    We write $u_{h,m}^n$ to denote the $n$th step of the exponential integrator method that uses a Krylov subspace of dimention $m$.
    We have that:
    \begin{align*}
        u_{h,m}^n &= e_m^{-\tau L_h} u_{h,m}^{n-1} + \tau \int^1_0e_m^{-\tau L_h(1-\theta)}d\theta P_hf(t, u_{h,m}^n)
    \end{align*}
\end{definition}