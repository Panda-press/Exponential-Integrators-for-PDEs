\section {Snowflakes}
Here we begin to look in the phase boundary of the formation of a snowflake.
We will use an adaptive grid and compaire the first and second order exponential integrator methods.
We will also compare the limitations of these methods for differnt krylov sizes.
\subsection{Formulation of problem}

We condider the problem on a domain $\Omega = \times [4, 8] \times [4, 8]$.
The equaitions govening this phase boundary problem are as follows.
\begin{align*}
    \tau \frac{\partial \phi}{\partial t} &= \nabla \cdot D\nabla\phi +  \phi(1-\phi)m(\phi,T)\\
    \frac{\partial T}{\partial t} &= D_T \nabla ^2T + \frac{\partial \phi}{\partial t}
\end{align*}

Where we have that

\begin{align*}
    m(\phi, T) &= \phi - \frac 12 - \frac{\kappa_1}{\pi}arctan(\kappa_2 T)\\
    D &= \alpha^2(1+c\beta)
    \begin{pmatrix}
        1+c\beta & -c\frac{\partial\beta}{\partial\psi}\\
        c\frac{\partial\beta}{\partial\psi} & 1+c\beta
    \end{pmatrix}\\
    \text{With }\beta &= \frac{1-\Phi^2}{1+\Phi^2}\\
    \Phi &= tan(\frac N2\psi)\\
    \psi &= \theta + arctan(\frac{\frac{\partial \phi}{\partial y}}{\frac{\partial \phi}{\partial x}})\\
\end{align*}
and with the following constants
\begin{align*}
    \begin{matrix}
    D_T = 2.25 & \alpha = 0.015 & \tau = 3\cdot 10^{-4} & \kappa_1 &= 0.9\\
    \kappa_2 = 20 & c = 0.02 & N = 6 & \theta = \frac{\pi}8
    \end{matrix}
\end{align*}

Substituting in $\frac{\partial \phi}{\partial t}$ into the second equation and rewriting in weak form gives the following
\begin{align*}
    \int \frac{\partial \phi}{\partial t} v_{\phi}dx &=  \frac 1{\tau}\int \phi(1-\phi)m(\phi,T)v_{\phi} - D \nabla \phi \cdot \nabla v_{\phi}dx\\
    \int \frac{\partial T}{\partial t} v_T dx &= \int \frac 1{\tau}(\phi(1-\phi)m(\phi,T)v_T - \nabla v_T \cdot D \nabla \phi) - D_T \nabla v_Tdx \cdot \nabla T
\end{align*}
Where we use homogeneus Neumann boundary condtions.
\begin{figure}[H]
    \centering
    \includegraphics[width=0.5\textwidth]{Snowflakes/EXP1LAN/Snowflake_2--problem_1.0_0_16.png} % Change filename to your image
    \caption{Initial condition}
    \label{fig:initial}
\end{figure}
With an initial condition (figure \ref{fig:initial}) of: 
\begin{align*}
    \phi(0,x,y) &= \begin{cases} 
      1 & (x - 6)^2 + (y - 6)^2 \leq 0.3 \\
      0 & \text{otherwise} \end{cases}\\
    T(0,x,y) &= -0.5 \text{  for  } (0,x,y)\in \Omega
\end{align*}

\subsection{Numerical Solutions}
Here we begin the comparison of the differnet methods.
For all of them we will use a time step size of $0.0005$ and will run untill and end time of $0.1$.\\

We begin looking at the results for the first and second order methods with a krylov subspace of dimension $16, 32$.

\begin{figure}[H]
    \centering
    \begin{minipage}{0.49\textwidth}
        \includegraphics[width=1\textwidth]{Snowflakes/EXP1LAN/Snowflake_2--problem_1.0_10_16.png} % Change filename to your image
        \caption{from first order exponential integrator with krylov size $16$}
        \label{fig:first order 16}
    \end{minipage}\hfill
    \centering
    \begin{minipage}{0.49\textwidth}
        \includegraphics[width=1\textwidth]{Snowflakes/EXP1LAN/Snowflake_2--problem_1.0_10_32.png} % Change filename to your image
        \caption{from first order exponential integrator with krylov size $32$}
        \label{fig:first order 32}
    \end{minipage}\hfill
\end{figure}\begin{figure}[H]
    \centering
    \begin{minipage}{0.49\textwidth}
        \includegraphics[width=1\textwidth]{Snowflakes/EXP2LAN/Snowflake_2--problem_1.0_10_16.png} % Change filename to your image
        \caption{from second order exponential integrator with krylov size $16$}
        \label{fig:second order 16}
    \end{minipage}\hfill
    \centering
    \begin{minipage}{0.49\textwidth}
        \includegraphics[width=1\textwidth]{Snowflakes/EXP2LAN/Snowflake_2--problem_1.0_10_32.png} % Change filename to your image
        \caption{from second order exponential integrator with krylov size $32$}
        \label{fig:second order 32}
    \end{minipage}\hfill
\end{figure}

We observe that the snowflakes generated by both krylov sizes are near identical, 
suggesting that setting the dimension of the krylov space to $16$ is sufficient for this problem.\\

\subsection{Limitations}
Now we compare the limitations of the first and second order exponential methods when the dimension of the krylov space is smaller that for what we saw above.
We compare it for krylov sizes of $8$ and $10$.

\begin{figure}[H]
    \centering
    \begin{minipage}{0.49\textwidth}
        \includegraphics[width=1\textwidth]{Snowflakes/EXP1LAN/Other/Snowflake_2--problem_1.0_5_8.png} % Change filename to your image
        \caption{from first order exponential integrator with krylov size $8$ at $t=0.05$}
        \label{fig:first order 8}
    \end{minipage}\hfill
    \centering
    \begin{minipage}{0.49\textwidth}
        \includegraphics[width=1\textwidth]{Snowflakes/EXP1LAN/Snowflake_2--problem_1.0_10_10.png} % Change filename to your image
        \caption{from first order exponential integrator with krylov size $10$}
        \label{fig:first order 10}
    \end{minipage}\hfill
\end{figure}
\begin{figure}[H]
    \centering
    \begin{minipage}{0.49\textwidth}
        \includegraphics[width=1\textwidth]{Snowflakes/EXP2LAN/Other/Snowflake_2--problem_1.0_5_8.png} % Change filename to your image
        \caption{from second order exponential integrator with krylov size $8$ at $t=0.05$}
        \label{fig:second order 8}
    \end{minipage}\hfill
    \centering
    \begin{minipage}{0.49\textwidth}
        \includegraphics[width=1\textwidth]{Snowflakes/EXP2LAN/Snowflake_2--problem_1.0_10_10.png} % Change filename to your image
        \caption{from second order exponential integrator with krylov size $10$}
        \label{fig:second order 10}
    \end{minipage}\hfill
\end{figure}

When a subspace of size $8$ is used, as shown in figure \ref{fig:first order 8},
we observe that the snowflake continuses to grow in the shape of a smooth hexagon without forming any dendrites.
This hexagon continued to grow untill it pressed up agains the boundaries of the domain.
When we use a subspace of size $10$ we observe that while dendrites now form but they are stubbier and produce less branching figure \ref{fig:first order 10}.

We now repeat the above but halving the step size and keeping the krylov size the same, at $8$ and $10$.

\begin{figure}[H]
    \centering
    \begin{minipage}{0.49\textwidth}
        \includegraphics[width=1\textwidth]{Snowflakes/EXP1LAN/Snowflake_2--problem_0.5_10_8.png} % Change filename to your image
        \caption{from first order exponential integrator with krylov size $8$}
        \label{fig:first order 8 0.5}
    \end{minipage}\hfill
    \centering
    \begin{minipage}{0.49\textwidth}
        \includegraphics[width=1\textwidth]{Snowflakes/EXP1LAN/Snowflake_2--problem_0.5_10_10.png} % Change filename to your image
        \caption{from first order exponential integrator with krylov size $10$}
        \label{fig:first order 10 0.5}
    \end{minipage}\hfill
\end{figure}
\begin{figure}[H]
    \centering
    \begin{minipage}{0.49\textwidth}
        \includegraphics[width=1\textwidth]{Snowflakes/EXP2LAN/Snowflake_2_0.5_10_8.png} % Change filename to your image
        \caption{from second order exponential integrator with krylov size $8$}
        \label{fig:second order 8 0.5}
    \end{minipage}\hfill
    \centering
    \begin{minipage}{0.49\textwidth}
        \includegraphics[width=1\textwidth]{Snowflakes/EXP2LAN/Snowflake_2_0.5_10_10.png} % Change filename to your image
        \caption{from second order exponential integrator with krylov size $10$}
        \label{fig:second order 10 0.5}
    \end{minipage}\hfill
\end{figure}

We see that even with a smaller krylov subspace we can generate plots similar to those that use a larger subspace so long as the time step is small enough.
This match with what we previously observed with the above Allen Cahn eqautions.