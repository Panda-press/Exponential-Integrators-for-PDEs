\section{Solution Schemes}

When solving the following ODE:
\begin{align*}
M\dot u_h(t) &= N_h(u_h(t))\\ %\text{ where $M$ is the mass matrix}
\text{using } R(u_h(t)) &= N_h(u_h(t)) - DN(u_h(t))u_h(t) \text{to denote the non-linear term gives}\\
M\dot u_h(t) &= DN(u_h(t))u_h(t) + R(u_h(t))\\
\dot u_h(t) &= M^{-1}(DN(u_h(t))u_h(t) + R(u_h(t)))
\end{align*}
\subsection{Methods}
For a time step $\tau$ we compute $u_h(t+\tau)$ in the following ways: 
\subsubsection{Backward Euler}
The backward euler method is given by:
\begin{align*}
u_h(t+\tau) = u_h(t) + \tau M^{-1}N_h(u_h(t+\tau))
\end{align*}
\subsubsection{Explicit Exponential Scheme} %check this is correct%
For the explicity exponential integrator schemes we use the following formular:
\begin{align*}
u_h(t+\tau) &= e^{\tau M^{-1} DN(u_h(t))}(u_h(t) + M^{-1}R(u_h(t)))
\end{align*}
where the action of the matrix exponential on a vector is computed according the the various Krylov methods.

\subsubsection{First order Exponential Integrator}
Here we also present another integrator from Huang Et al \cite{Huang2022}
\begin{align*}
u_h(t+\tau) &= e^{\tau M^{-1} DN(u_h(t))}u_h(t) + \tau \varphi_1(\tau M^{-1} DN(u_h(t)))R(u_h(t))
\end{align*}
Where we have that:
\begin{align*}
    \varphi_k(z) &= \int^1_0e^{(1-\theta)z}\frac{\theta^{k-1}}{(k-1)!}d\theta, k \geq 1
\end{align*}

Notice that the following holds:
\begin{align*}
    \varphi_k(A)v &= V\varphi_k(H)||v||e_1\\
\end{align*}
where $V,H$ are generated by one of the Krylov Methods above. 
We demonstrate this below:
\begin{align*}
    \varphi_k(A)v &= \int^1_0e^{(1-\theta)A}\frac{\theta^{k-1}}{(k-1)!}d\theta v\\
    &= \int^1_0e^{(1-\theta) A}d\theta v\\
    &= \int^1_0e^{(1-\theta) A}vd\theta\\
\intertext{applying a Krylov method giving $VH||v||e_1 \approx Av$ we get}
    &= \int^1_0Ve^{(1-\theta) H}||v||e_1d\theta\\
    &= V\int^1_0e^{(1-\theta) H}d||v||e_1\\
    &= V\varphi_k(H)||v||e_1
\end{align*}
As a result of this we will only need to genarate a single subspace for the numerical integrations.

\subsubsection{Second Order Exponential Integrator}
We also look at a second order exponential integrator \cite{Huang2022}.
\begin{align*}
    \text{writing } A &= \tau M^{-1} DN(u_h(t)) \text{ for the sake of brevity}\\
    u_h(t+\tau) &= e^{A}u_h(t) + \tau((\varphi_1(A)) - \frac 1{c_2}\varphi_2(A))R(u_h(t))\\
    & + \frac1{c_2}\varphi_2(A)R(e^{c_2A}u_h(t) + c_2\tau\varphi_1(c_2 A)R(u_h(t)))
\end{align*}
Where $c \in (0,1]$.\\
It should also be noted that the Krylov space used to compute $(\varphi_1(A))R(u_h(t))$ is the same subspace used to $\tau\varphi_1(c_2 A)R(u_h(t))$.
This is also true for the computation of $e^{A}u_h(t)$ and $e^{c_2A}u_h(t)$.

\subsection{Mass Matrix}
When using these schemes it is neccessary to be able to compute the inverse mass matrix $M^{-1}$.
Attempting to compute the exact inverse can be computationally intensive and while for a fixed discretisations this may be acceptable, as it will only need to be computed once, for an adaptive grid this may be impractical.
As a result we employ mass lumping where each row is summed up and placed on the diagonal of the matrix.
From here computing the inverse is straight forward.


