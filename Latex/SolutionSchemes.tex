\section{Solution Schemes} \label{section:methods}
Now, we outline the methods by which the ODEs arising from the discretization of PDEs in space can be solved. 
Consider the following PDE:
\begin{align*}
    \dot u_t &= N(u)
\end{align*}
Discretizing in space gives the following:
\begin{align*}
M\dot u_h(t) &= N_h(u_h(t))\\ %\text{ where $M$ is the mass matrix}
\text{using } R(u_h(t)) &= N_h(u_h(t)) - DN(u_h(t))u_h(t) \text{ to denote the non-linear term gives}\\
M\dot u_h(t) &= DN(u_h(t))u_h(t) + R(u_h(t))\\
\dot u_h(t) &= M^{-1}(DN(u_h(t))u_h(t) + R(u_h(t)))
\end{align*}
\subsection{Methods}
We present both conventional ODE solvers such as the forwards and backwards Euler method before moving on to the exponential integrator methods.
For a time step $\tau$ we compute $u_h(t+\tau)$ in the following ways: 

\subsubsection{Forwards Euler}
The forwards Euler method is given by:
\begin{align*}
u_h(t+\tau) = u_h(t) + \tau M^{-1}N_h(u_h(t))
\end{align*}

\subsubsection{Backwards Euler}
The backwards Euler method is given by:
\begin{align*}
u_h(t+\tau) = u_h(t) + \tau M^{-1}N_h(u_h(t+\tau))
\end{align*}
As this method is implicit, we will need to employ a Newton solver.

\subsubsection{Explicit Exponential Scheme} %check this is correct%
For the explicit exponential integrator schemes we use the following formula:
\begin{align*}
u_h(t+\tau) &= e^{\tau M^{-1} DN(u_h(t))}(u_h(t) + M^{-1}R(u_h(t)))
\end{align*}


\subsubsection{First Order Exponential Integrator}
Here, we also present another integrator from Huang Et al \cite{Huang2022}
\begin{align*}
u_h(t+\tau) &= e^{\tau M^{-1} DN(u_h(t))}u_h(t) + \tau \varphi_1(\tau M^{-1} DN(u_h(t)))R(u_h(t))
\end{align*}
Where:
\begin{align*}
    \varphi_k(z) &= \int^1_0e^{(1-\theta)z}\frac{\theta^{k-1}}{(k-1)!}d\theta, k \geq 1
\end{align*}

\subsubsection{Second Order Exponential Integrator}
We also look at a second order exponential integrator \cite{Huang2022}.
\begin{align*}
    \text{writing } A &= \tau M^{-1} DN(u_h(t)) \text{ for the sake of brevity}\\
    u_h(t+\tau) &= e^{A}u_h(t) + \tau((\varphi_1(A)) - \frac 1{c}\varphi_2(A))R(u_h(t))\\
    & + \frac1{c}\varphi_2(A)R(e^{cA}u_h(t) + c\tau\varphi_1(c A)R(u_h(t)))
\end{align*}
where $c \in (0,1]$.

\subsection{Mass Matrix}
When using these schemes it is necessary to be able to compute the inverse mass matrix $M^{-1}$.
Attempting to compute the exact inverse can be computationally intensive. While for a fixed spacial discretization this may be acceptable, as it will only need to be computed once, for an adaptive grid this may be impractical.
As a result, we employ mass lumping where each row is summed up and placed on the diagonal of the matrix.
From here, computing the inverse is straightforward.
