\section{Double Slit Wave Model}

In this section, we look into applying these exponential methods to the computation of solutions to the wave equation.
While the problem is linear, we will need to rewrite the problem as a system of equations that are first order with respect to time.
In doing so, the matrix will be antisymmetric.
This prohibits the use of the Lanczos algorithm, requiring Arnoldi to be used, which is more expensive.
Specifically, we will look into solutions to the double slit wave problem:

\subsection{Formulation of Problem}
Where our domain is given by $\Omega$ as seen in figure \ref{fig:waveOmega}, we write the problem as:
\begin{align*}
    \psi_{tt} &= \nabla \psi \quad x \in \Omega, 0<t\leq T
    \psi(0,x) = 0 \quad x \in \Omega
\end{align*}
where $T=3$ is our end time for the simulation.
Let $S_L = \{0\} \times [0,1] \subset \mathbb{R}^2$ denote the left side of the domain $\Omega$.
We implement the following Dirichlet boundary condition:
\begin{align*}
    \psi &= \frac{1}{10\pi} sin(10\pi t) x \in S_L
\end{align*}
This will be the source of our waves.
The remaining boundaries use homogeneous Neumann boundary conditions.
Our methods, both exponential and backwards Euler, in section \ref{section:methods}, require a first order equation with respect to the derivatives of $t$.
This means that we will need to use a substitution.
We can use $\psi_t = -p$ to get the following:
\begin{align*}
    \psi_t &= -p\\
    p_t &= \nabla \psi\\
\end{align*}
with
\begin{align*}
    p(0,x,y) &= 0
\end{align*}
This equation is now compatible with the methods that we are using.

\subsection{Numerical Solutions}
We can now compare backwards Euler and the two exponential integrator methods.
First, we show the domain as well as a solution computed with the backwards Euler method.
For the backwards Euler, a timestep of $\tau = 0.001$ was found to be optimal, with larger timesteps causing numerical issues, leading to instability, and smaller timesteps requiring excessive computation.

\begin{figure}[H]
    \centering
    \begin{minipage}{0.49\textwidth}
        \includegraphics[width=1\textwidth]{Wave/DoubleSlit_0_1.0_0_64.png} % Change filename to your image
        \caption{Domain $\Omega$ with grid displayed}
        \label{fig:waveOmega}
    \end{minipage}\hfill
    \centering
    \begin{minipage}{0.49\textwidth}
        \includegraphics[width=1\textwidth]{Wave/BE/DoubleSlit_0_0.1_10_64.png} % Change filename to your image
        \caption{Solution at time $T=3$ computed using the backwards Euler method}
        \label{fig:second order 32}
    \end{minipage}\hfill
\end{figure}

We now compare the backwards Euler to the first and second order exponential integrator methods.
Note that, due to the fact that the linear term given by $DN(u)$ is not symmetric, we cannot use the Lanczos method for computing our matrix exponential.
As a result, the Arnoldi algorithm for the matrix exponential has been used here exclusively.
\begin{figure}[H]
    \centering
    \begin{minipage}{0.49\textwidth}
        \includegraphics[width=1\textwidth]{Wave/EXP1ARN/DoubleSlit_0_2.0_10_24.png} % Change filename to your image
        \caption{Solution at time $T=3$ computed using the first order exponential integrator with timestep $\tau = 0.02$ and a Krylov size of $24$}
        \label{fig:second order 16}
    \end{minipage}\hfill
    \centering
    \begin{minipage}{0.49\textwidth}
        \includegraphics[width=1\textwidth]{Wave/EXP2ARN/DoubleSlit_0_2.0_10_24.png} % Change filename to your image
        \caption{Solution at time $T=3$ computed using the second order exponential integrator with timestep $\tau = 0.02$ and a Krylov size of $24$}
        \label{fig:second order 32}
    \end{minipage}\hfill
\end{figure}

We see that the first and second order exponential methods using the Arnoldi algorithm are capable of producing results equivalent to that of the backwards Euler method.

\subsection{Performance}
We now observe the performance differences between backwards Euler and the exponential integrators.
The number of calls to the operator is again used in order to measure performance.
We present the methods used and the operator calls in the table below:

\begin{table}[H]
    \centering
    \begin{tabular}{| c | c | c | c |}
    \hline
    Method & Required $\tau$ & Krylov size & Operator Calls\\
    \hline
    Backwards Euler & 0.001 & - & 119362 \\
    First Order Exponential & 0.02 & 24 & 7800 \\
    Second Order Exponential & 0.02 & 24 & 12300 \\
    \hline
    \end{tabular}
    \caption{Performance of Various Methods}
    \label{tab:reduced_data}
\end{table}

We observe that the backwards Euler method requires significantly more calls to the operator than both the first and second order exponential methods.
The second order method requires more calls to the operator than the first order method.
Both exponential methods required a similar timestep size $\tau$.
This suggests that the exponential methods may be effective for computing solutions to the wave equation.
This is significant as it shows the possible application of these methods to problems beyond parabolic PDEs.