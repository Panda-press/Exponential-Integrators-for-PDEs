\section{Double Slit Wave Model}

In this section, we look into applying these exponential steppers to the computation of solutions to the wave eqaution.
We will look into solutions the the double slit wave problem:

\subsection{Formulation of the Problem}
Where our domain is given by $\Omega$, we write the problem as:
\begin{align*}
    \psi_{tt}\psi &= \nabla \psi \quad x \in \Omega, 0<t\leq T\\
    \psi &= \frac{1}{10\pi} sin(10\pi t) x \in S_L
\end{align*}
where $T=3$ is our end time for the simulation and $S_L$ denotes the left side of the domain $\Omega$ this will be the source of our waves.
We can use $\psi_t = -p$ to get the following:
\begin{align*}
    \psi_t &= -p\\
    p_t &= \nabla \psi\\
    \psi &= \frac{1}{10\pi} sin(10\pi t) x \in S_L
\end{align*}
This equation is now compatable with the time steppers that we are using.

\subsection{Numerical Solutions}
We can now compare backwards euler and the two exponential integrator methods.
First we show the the domain as well as a solution computed with the backwards Euler method.
For the backwards Euler a time step of $\tau = 0.001$ was found to be optimal, with larger timesteps causing numerical issues and smaller timesteps requiring excessive computation.

\begin{figure}[H]
    \centering
    \begin{minipage}{0.49\textwidth}
        \includegraphics[width=1\textwidth]{Wave/DoubleSlit_0_1.0_0_64.png} % Change filename to your image
        \caption{Domain $\Omega$ with grid displayed}
        \label{fig:second order 16}
    \end{minipage}\hfill
    \centering
    \begin{minipage}{0.49\textwidth}
        \includegraphics[width=1\textwidth]{Wave/BE/DoubleSlit_0_0.1_10_64.png} % Change filename to your image
        \caption{Solution at time $T=3$ computed using the backwards euler method}
        \label{fig:second order 32}
    \end{minipage}\hfill
\end{figure}

We now compare this to the first and second order exponential integrator methods.
Note that due the fact that the linear term given by $DN(u)$ is not symetric we cannot use the Lancoz method for computing our matrix exponential.
As a result the Arnoldi alorithm for the matrix exponential has been used here exculively.
\begin{figure}[H]
    \centering
    \begin{minipage}{0.49\textwidth}
        \includegraphics[width=1\textwidth]{Wave/EXP1ARN/DoubleSlit_0_2.0_10_24.png} % Change filename to your image
        \caption{Solution at time $T=3$ computed Using the first order exponential integrator with timestep $\tau = 0.02$ and a Krylov size of $24$}
        \label{fig:second order 16}
    \end{minipage}\hfill
    \centering
    \begin{minipage}{0.49\textwidth}
        \includegraphics[width=1\textwidth]{Wave/EXP2ARN/DoubleSlit_0_2.0_10_24.png} % Change filename to your image
        \caption{Solution at time $T=3$ cmoputed using the second order exponential integrator with timestep $\tau = 0.02$ and a Krylov size of $24$}
        \label{fig:second order 32}
    \end{minipage}\hfill
\end{figure}

We see that the first and second order exponential methods using the Arnoldi algorithm are capable of producing results equivalent to that of the backwards Euler method.

\subsection{Performance}
We now observe the performance differences between the methods.
For this, the number of calls to the operator is again used in order to measure performance.
We present the methods used and the operator calls in the table below:

\begin{table}[H]
    \centering
    \begin{tabular}{| c | c | c | c |}
    \hline
    Method & Required $\tau$ & Krylov size & Operator Calls\\
    \hline
    Backwards Euler & 0.001 & - & 119362 \\
    First Order Expoenntial & 0.02 & 24 & 7800 \\
    Second Order Expoenntial & 0.02 & 24 & 12300 \\
    \hline
    \end{tabular}
    \caption{Performance of Various Methods}
    \label{tab:reduced_data}
\end{table}

We observe that Backwards Euler method requires significantly more calls to the operator that both the first and second order exponential methods.
The second order method requires more calls to the operator than the first order method.
This suggest that these methods may be effective for computing solutions to the wave equation.