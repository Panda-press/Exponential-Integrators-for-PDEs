\section{Numerical Experiments}

In this section we run numerical experiments compairing the backward euler methods to Kyrlov methods for computing solutions to Allen Cahn equation with a traveling wave solution\cite{YukitakaFukao2004}:

\begin{align}
    \hat u(x,y,t)=\frac{e^{\frac{x-ct}{\sqrt2}}}{1+e^{\frac{x-ct}{\sqrt2}}} \label{TravelingWaveSol}
\end{align}
and where the porblem is defined as follows:
\begin{align*}
    u_t&=\Delta u+u(u-\frac14)(1-u)\\
    \text{with } \Omega &= \mathbb{R}\times[-1,1]\\
    u(x,y,t) &= \hat u(x,y,t) \text{ for $(x,y,t) \in \mathbb{R}\times\{-1,1\}\times\mathbb{R}_+$}\\
    u_0 &= \hat u(x,y,0)\\
\end{align*}
However for our numerical experiments we compute on the space $\Omega=[-8,8]\times[-1,1]$.
Which by enforcing the boundarys so that they match the analytic soltuion it gives us the following boundary condionts:
\begin{align*}
    u(x,y,t) &= \hat u(x,y,t) \text{ for $(x,y,t) \in \mathbb{R}\times\{-1,1\}\times\mathbb{R}_+$}\\
    u(x,y,t) &= \hat u(x,y,t) \text{ for $(x,y,t) \in \{-4,8\}\times[-1,1]\times\mathbb{R}_+$}
\end{align*}
We now procede to write the problem in its weak formulation:
\begin{align*}
    \dot u&=\Delta u+u(u-\frac14)(1-u)\\
    \int_{\Omega} \dot u v &=\int_{\Omega} \Delta uv+u(u-\frac14)(1-u)v\\
    \int_{\Omega} \dot u v &=\int_{\Omega} u(u-\frac14)(1-u)v - \nabla u \cdot \nabla v + \int_{\partial\Omega}  v\hat n \cdot \nabla u\\
    \text{subsituting in } u_h(t,x,y) &= \sum_i u_i(t) v_i(x,y)\\
    \text{we get } \int_{\Omega} \dot u_i v_i v_j &=\int_{\Omega} u_iv_i(u_iv_i-\frac14)(1-u_iv_i)v_j - \nabla (u_iv_i) \cdot \nabla v_j + \int_{\partial\Omega}  v_j\hat n \cdot \nabla u_iv_i\\
    u_i\int_{\Omega}v_iv_j &= -u_i\int_{\Omega}\nabla v_i \cdot \nabla v_j + \int_{\Omega}R(u_h)v_j\\
    \text{where we have } R(u)&=u(u-\frac14)(1-u) + \int_{\partial\Omega}  \hat n \cdot \nabla u\\
    \intertext{writing in matrix form gives}
    M\dot u_h &= DN(u_h) + R(u_h)\\
\end{align*}

In these experiments we use the number of calls to the operator as an approximate estimate of compute cost and compare this to the $L_2$ error when compared to the analytic solution \eqref{TravelingWaveSol}.
We compare the Backward Euler, Arnoldi, Lanczos and Kiops methods over grid sizes: $to be decided$ and with time step $\tau=tobedecided$.

The boundary condition are enforced such that they match the analytic solution.

\subsection{Results}

Bellow we show the error with respect to the number of calls to the operator $N$ which are needed both for computing $DN$ via the matrix free method described above as well as $R$.


First ovbserve as expected that when are larger krylov subspace is used that more calls to the operator are provided.


