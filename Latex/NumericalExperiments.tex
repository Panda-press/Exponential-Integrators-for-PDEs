\section{Numerical Experiments}

In this section we run numerical experiments compairing the backward Euler method to krylov methods for computing solutions to a PDE.
Specifically we will investigate the initial boundary value problem for the Allen Cahn equation on a strip with a traveling wave solution\cite{YukitakaFukao2004} given by:

\begin{align}
    \hat u(x,y,t)&=\frac{e^{\frac{x-ct}{\sqrt2}}}{1+e^{\frac{x-ct}{\sqrt2}}} \label{TravelingWaveSol}\\
    \text{where } c &= \frac{\sqrt{2}}{4}
\end{align}
and where the problem is defined as follows:
\begin{align*}
    u_t&=\Delta u+u(u-\frac14)(1-u) \text{ on $\Omega \times [0, 8]$}\\
    \text{with } \Omega &= \mathbb{R}\times[-1,1]\\
    \nabla u \cdot \hat n &= g \text{ for $g = \nabla \hat u \cdot \hat n$}\\
    u_0(x, y) &= \hat u(x,y, 0)\\
\end{align*}
However for our numerical experiments we compute on the space $\Omega=[-16,16]\times[-1,1]$.
Where $u$ is the solution, we write it in weak form.
\begin{align*}
    \dot u&=\Delta u+u(u-\frac14)(1-u)\\
    \int_{\Omega} \dot u v &=\int_{\Omega} \Delta uv+u(u-\frac14)(1-u)v \text{ for a smooth function $v \in C^{\infty}(\overline{\Omega})$}\\
    \int_{\Omega} \dot u v &=\int_{\Omega} u(u-\frac14)(1-u)v - \nabla u \cdot \nabla v + \int_{\partial\Omega}  v\hat n \cdot \nabla u\\
    \text{subsituting in } u_h(t,x,y) &= \sum_i u_i(t) v_i(x,y) \text{ where $(v_i)$ are the test functions we obtain:}\\
    \text{we get } \sum_i\int_{\Omega} \dot u_i v_i v_j &=\sum_i(\int_{\Omega} u_iv_i(u_iv_i-\frac14)(1-u_iv_i)v_j - \int_{\Omega}\nabla (u_iv_i) \cdot \nabla v_j + \int_{\partial\Omega}  v_j\hat n \cdot \nabla \hat u_iv_i)\\
    \sum_iu_i\int_{\Omega}v_iv_j &= -u_i\sum_i(\int_{\Omega}\nabla v_i \cdot \nabla v_j + \int_{\Omega}R(u_h)v_j)\\
    \text{where we have } R(u)&=u(u-\frac14)(1-u) + \int_{\partial\Omega}  \hat n \cdot \nabla \hat  u\\
    \intertext{The above is true for all $i=0,1,2,...,n$. Writing in matrix vector form gives}
    M \underline{\dot u} &= DN(\underline{u})\underline{u} + R(\underline{u})\\
    \text{where } M_{i,j} &= \int_{\Omega}v_iv_jdx
\end{align*}
In the tests, we will investigate the relation between step size, $L_2$ error given by $||\hat u - \underline{u}||_{L_2}$ and call to the operator $N$ which will be a crude approximation to the computational cost.
We compare the Backward Euler, First and Second order expoenntial methods over grid sizes: $128 \times 8$ and $256 \times 8$ and with time step $\tau=8,4,2,1,0.5,0.25,0.125,0.0625,0.03125$.

\subsection{experimental Results}

Bellow we show the $L_2$ error with respect to the time step $\tau$.
Where in the legend EXP1LAN 16 denotes the first order method with Krylov size 16, likewise for EXP2LAN and the second order method.
BE denotes the backwards Euler method.

\begin{figure}[H]
    \centering
    \begin{minipage}{0.49\textwidth}
        \includegraphics[width=1\textwidth]{Graphs/TravellingWave/Tau V L2 Error for Travelling Wave with grid size 1024.png} % Change filename to your image
        \caption{$\tau$ vs $L_2$ with grid size 1024}
        \label{fig:plot1}
    \end{minipage}\hfill
    \centering
    \begin{minipage}{0.49\textwidth}
        \includegraphics[width=1\textwidth]{Graphs/TravellingWave/Tau V L2 Error for Travelling Wave with grid size 2048.png} % Change filename to your image
        \caption{$\tau$ vs $L_2$ with grid size 2048}
        \label{fig:plot2}
    \end{minipage}\hfill
\end{figure}

First see that all of the methods tested converge, tending towards the discritisation error.
The discritisation error is smaller on the more detailed grid.
The Krylov methods converge faster than the backwards Euler method.
We notice that when a smaller Krylov subspace is used that the expoenntial methods only start to converge for a sufficiently small tau.
Likewise when these methods start to converge we observe that the perform simillarly to the methods using a larger subspace, however they will require fewer operator calls.
This suggests that for sufficient small time steps a shallower subspace may suffice.
Likewise for the converse that larger timesteps require a deeper krylov subspace.

Now we compare the error with the number of calls to the operator.
These operator calls are needed for computing $N$ which is used with the matrix free methods as described above as well as for computing the non-linear term $R$.

\begin{figure}[H]
    \centering
    \begin{minipage}{0.49\textwidth}
        \includegraphics[width=1\textwidth]{Graphs/TravellingWave/Operator Calls V Error for Travelling Wave with grid size 1024.png} % Change filename to your image
        \caption{$\tau$ vs $L_2$ with grid size 1024}
        \label{fig:plot1}
    \end{minipage}\hfill
    \centering
    \begin{minipage}{0.49\textwidth}
        \includegraphics[width=1\textwidth]{Graphs/TravellingWave/Operator Calls V Error for Travelling Wave with grid size 1024.png} % Change filename to your image
        \caption{$\tau$ vs $L_2$ with grid size 2048}
        \label{fig:plot2}
    \end{minipage}\hfill
\end{figure}

We see that as expected that when are larger krylov subspace is used that more calls to the operator are required.
We also observe that the exponential methods outperform the BE by achieving a lower error with fewer calls to the operator.

Here we present the experimental orders of convergence.

\begin{table}[H]
    \centering
    \begin{tabular}{| c | c | c | c}
    \hline
    BE & EXP1LAN 64 & EXP2LAN 64 \\
    \hline
    1.656175178817219 & 2.5200121901844423 & 2.427488988868435 \\
    1.159963039219372 & 2.147211103525317 & 2.2448741639287576 \\
    1.0543798412512431 & 2.1084233436748687 & 2.175268892119002 \\
    1.0188791686457326 & 2.235535140373336 & 2.176237872787299 \\
    1.0016513655730614 & 2.2847781778148475 & 0.6355415381730404 \\
    0.9874831778164389 & -0.3463812590562603 & -0.1610890501147955 \\
    0.9808216274859436 & -0.2097197508087928 & -0.0545528584432596 \\
    0.919336864575657 & -0.051644372942987 & -0.01407844052927163 \\
    \hline
    \end{tabular}
    \caption{Reduced Data Table}
    \label{tab:reduced_data}
\end{table}


We observe that the rate of convergence for the second order method was inline with what has been shown analytic\cite{Huang2022} at a rate of $2$.
The first order method converges with a rate of 2 despite an expectation of a rate of $1$\cite{Huang2022}.
The exponential methods stop converging as they reach the discritisation error.
The backwards euler converges at the rate expceted of 1.