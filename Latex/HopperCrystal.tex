\section{3D Crystal}
We now investigate the modeling of a 3D crystal.
\subsection{Formulation of Problem}
We employ the phase field model as described by \cite{Bollada2023}.
\begin{align*}
    \dot \phi &= \nabla \cdot \frac{\partial}{\partial \nabla \phi}(\frac12 A^2)-\frac{\Omega'(\phi)}{\delta^2} - \frac{g'(\phi)(\mu_0-\mu)\Delta c}{\lambda \delta^2}\\
    \dot \mu &= a\nabla \cdot D \nabla \mu - a \Delta c g'(\phi)\dot \phi\\
    \phi(0,x) &= \phi_{t=0}\\
    \mu(0,x) &= \mu_{t=0}
\end{align*}
for some initial conditions $\phi_{t=0}$ and $\mu_{t=0}$ and with an anisotropy given by $A$.
\begin{align*}
    \phi_{t=0} &= \frac{1}{1+e^{\frac{-\sqrt{x^2 + y^2 + z^2}-R_0}{\delta}}}\\
    \mu_{t=0} &= \phi_{t=0} - \phi_{t=0}a(\bar c - c_s)
\end{align*}
The constants are given by:
\begin{align*}
    \begin{matrix}
    c_L = 0.9 & c_S = 0.5 & \mu_0 = 1 & \mu_{\inf} = 0.04 & a = 4 & D_L= \frac{1}{12} & D_S = 10^{-4}D_L\\
    \Delta c = c_L - c_S & \lambda = \frac{3R_c\Delta c^2}{\delta} & R_c = 10 & R_0 = 20 & \delta = 2 & \epsilon = 0.02\\
    \end{matrix}
\end{align*}

We will observe two different anisotropies.
The first a "cubic" anisotropy given by:
\begin{align*}
    A &= \sum_i \sqrt{\frac{\partial \phi}{\partial x^i}^2 + \epsilon^2(\frac{\partial \phi}{\partial x^1}^2 + \frac{\partial \phi}{\partial x^2}^2 + \frac{\partial \phi}{\partial x^3}^2)}
\end{align*}
and an octohedral anisotropy given by \cite{Bollada2015}:
\begin{align*}
    A &= A_0 (1 + \bar \epsilon(n^4_x+n^4_y+n^4_z))
\end{align*}
where we have that $A_0 =1 -3\epsilon$ and $\bar \epsilon = \frac{4\epsilon}{1-3\epsilon}$.


\subsection{Numerical Solutions}
For both these methods, we were limited by computing power.
Additional computing power would be required to properly run these simulations with the required temporal and spacial discretization to achieve accurate results.
The orginal simulations took 10s of hours on workstation computers.
Likely, due to issues with optimisation and parallelism, we were unable to properly scale to the required size.
As a result, we present solutions only after a smaller number of timesteps, in order to provide a limited demonstration of these methods working in three dimensions and to show the effect of different anisotropies on the formation of crystals.
Here, we present the numerical solution with the cubic anisotropy at time $t=400$.
We present the isosurfaces for $\phi = 0.5$.
For this, we use an adaptive grid mesh and $\tau = 0.1$.
\begin{figure}[H]
    \centering
    \begin{minipage}{0.49\textwidth}
        \includegraphics[width=1\textwidth]{Hopper/Square Hopper Crystal EXP1ARN.png} % Change filename to your image
        \caption{from first order exponential integrator with Krylov size $16$}
        \label{fig:first order 8 0.5}
    \end{minipage}\hfill
    \centering
    \begin{minipage}{0.49\textwidth}
        \includegraphics[width=1\textwidth]{example-image-b} % Change filename to your image
        \caption{from second order exponential integrator with Krylov size $16$}
        \label{fig:first order 10 0.5}
    \end{minipage}\hfill
\end{figure}
We observe that the crystal seed has now formed a cubic shape.
Had the simulation been able to run for longer, with a smaller time size, we would expect the edges and corners to protrude outwards forming a Hopper crystal.


Here, we present the numerical solution with the octohedral anisotropy at time $t=20$.
For this, we again, use an adaptive grid mesh and $\tau = 0.01$.
\begin{figure}[H]
    \centering
    \begin{minipage}{0.49\textwidth}
        \includegraphics[width=1\textwidth]{Hopper/Oct Hopper EXP1ARN.png} % Change filename to your image
        \caption{from first order exponential integrator with Krylov size $16$}
        \label{fig:first order 8 0.5}
    \end{minipage}\hfill
    \centering
    \begin{minipage}{0.49\textwidth}
        \includegraphics[width=1\textwidth]{example-image-b} % Change filename to your image
        \caption{from second order exponential integrator with Krylov size $16$}
        \label{fig:first order 10 0.5}
    \end{minipage}\hfill
\end{figure}
We observe that the crystal seed has now formed into the expected shape.
Along the edges and corners of the shape, numerical instability can be observed.
This instability aligns with the width of the spacial discretization.
We would expect dendrites to pertrude outwards from the corners after a longer simulation.

