\section{Parallelization}

Here we investigate how well the model can be parallelized.
We will investigate this for both the travelling wave problem as well as for modeling 2D snowflake growth.
For both, we will investigate how these mehtods scale on different numbers of cores for differnet Krylov subspaces and measure timings accordingly.
These methods will be compared to the backwards Euler method.

\subsection{Travelling Wave Allen Cahn}

We will compare the first and second order methods for a variety of gridsizes and Krylov subspace sizes.
The grid sizes to be used are as follows: to be decided and the Krylov sizes will be: to be decided.


\subsection{2D Snowflake Modeling}

Here we look at how well the Krylov methods scale in relation to Krylov subspace size using subspaces of dimension ~.
The grid complexity is govened by adaptive mesh refinement.

\begin{table}[H]
    \centering
    \begin{tabular}{| c | c c | c c | c c | c c | c c | c c |}
    \hline
    Cores & \multicolumn{2}{c|}{BE} & \multicolumn{2}{c|}{EXP1LAN 16} & \multicolumn{2}{c|}{EXP1LAN 32} & \multicolumn{2}{c|}{EXP2LAN 16} & \multicolumn{2}{c|}{EXP2LAN 32} \\
    \hline
    & T & CR & T & CR & T & CR \\
    \hline
    1  & 0 & 0 & 0 & 0 & 0 & 0 \\
    2  & 0 & 0 & 0 & 0 & 0 & 0 \\
    4  & 0 & 0 & 0 & 0 & 0 & 0 \\
    8  & 0 & 0 & 0 & 0 & 0 & 0 \\
    16 & 0 & 0 & 0 & 0 & 0 & 0 \\
    \hline
    \end{tabular}
    \caption{Reduced Data Table}
    \label{tab:reduced_data}
\end{table}
