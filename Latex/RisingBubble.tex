\section{Rising Bubble}

Here we will look into computing a rising bubble, in order to investigate how well these methods work for problems in the modeling of atmospheric processes.
This also helps to demonstrate whether these methods can handle advection.
This problem will be modeled on a finite volume space.
This will allow us to investigate whether or not the performance improvements are limited to finite element approximations.
\subsection{Formulation of Problem}
The problem is given by the following equations\cite{Bryan2002}:
\begin{align*}
    \frac{\partial u_i}{\partial t} &= -\frac{\partial u_i u_j}{\partial x_j} + u_i\frac{\partial u_j}{\partial x_j} - c_p \theta \frac{\partial \pi}{\partial x_i} \\
    \frac{\partial \pi}{\partial t} &= -\frac{\partial u_j \pi}{\partial x_j} + \pi\frac{\partial u_j}{\partial x_j} - \pi \frac{R}{c_v}\frac{\partial u_j}{\partial x_j} \\
    \frac{\partial \theta}{\partial t} &= -\frac{\partial u_j \theta}{\partial x_j} + \theta\frac{\partial u_j}{\partial x_j}
\end{align*}
where Einstein's summing notation is used.
We use a domain $\Omega = (0,1000) \times (0,2000)$.
The constants are given by:
\begin{align*}
    \begin{matrix}
    c_p = 1005 & c_v = 717.95 \\
    R = c_p - c_v 
    \end{matrix}
\end{align*}

We enforce reflective boundary conditions along the sides of the domain.
The initial condtions are as follows:
\begin{align*}
    u_i &= 0\\
    \pi(0,x,y) &= \pi_0\\
    \theta &= 0
\end{align*}
where $\pi_0$ is shown below
\begin{figure}[H]
    \centering
    \begin{minipage}{1\textwidth}
        \includegraphics[width=1\textwidth]{Bubble/initial.png} % Change filename to your image
        \caption{Initial Condition of $\pi_0$}
        \label{fig:first order 8 0.5}
    \end{minipage}
\end{figure}
\subsection{Numerical Solutions}
We run our simulations on a finite volume mesh.
The grid is rectangular and of dimension $64\times 128$.
We use a timestep of $\tau = 1.8$ as this was optimal when used with the backwards Euler method.
We plot the value of $\pi$ at time $t=4000$.
\begin{figure}[H]
    \centering
    \begin{minipage}{0.49\textwidth}
        \includegraphics[width=1\textwidth]{Bubble/EXP1ARN32.png} % Change filename to your image
        \caption{from first order exponential integrator with Krylov size $32$}
        \label{fig:first order 8 0.5}
    \end{minipage}\hfill
    \centering
    \begin{minipage}{0.49\textwidth}
        \includegraphics[width=1\textwidth]{Bubble/EXP1ARN64.png} % Change filename to your image
        \caption{from first order exponential integrator with Krylov size $64$}
        \label{fig:first order 10 0.5}
    \end{minipage}\hfill
\end{figure}
\begin{figure}[H]
    \centering
    \begin{minipage}{0.49\textwidth}
        \includegraphics[width=1\textwidth]{Bubble/EXP2ARN32.png} % Change filename to your image
        \caption{from second order exponential integrator with Krylov size $32$}
        \label{fig:first order 8 0.5}
    \end{minipage}\hfill
    \centering
    \begin{minipage}{0.49\textwidth}
        \includegraphics[width=1\textwidth]{Bubble/EXP2ARN64.png} % Change filename to your image
        \caption{from second order exponential integrator with Krylov size $64$}
        \label{fig:first order 10 0.5}
    \end{minipage}\hfill
\end{figure}
For Krylov sizes above $64$ the computed solution didn't change.
The solution computed with a Krylov size of $64$ was the same as the solution by the backwards Euler method as well as for smaller $\tau$.
We therefore conclude that this is close to the true solution.

\subsection{Performance}
We now observe the performance differences between backwards Euler and the exponential integrators.
The number of calls to the operator is again used in order to measure performance.
We present the methods used and the operator calls in the table below:

\begin{table}[H]
    \centering
    \begin{tabular}{| c | c | c | c |}
    \hline
    Method & Krylov size & Operator Calls\\
    \hline
    Backwards Euler & - & 156879 \\
    First Order Exponential & 32 & 151640 \\
    First Order Exponential & 64 & 303280 \\
    Second Order Exponential & 32 & 227460 \\
    Second Order Exponential & 64 & 454920 \\
    \hline
    \end{tabular}
    \caption{Performance of Various Methods}
    \label{tab:reduced_data}
\end{table}

We see that these exponential integrator methods required more operator calls than the backwards Euler method to achieve comparable results for this timestep size.


Further study could include finding the optimal tau for these methods that minimises the number of operator calls.
It would also make sense to explore the possible values of $m$ in between 32 and 64 as we would expect to find that these methods converge to the expceted solution for a Krylov size between these two values.
We would also like to investigate how these methods perform for discontinuous galerkin spaces for the rising bubble problem.
However, due to time constraints on the project, it has not been possible to gather this information.

