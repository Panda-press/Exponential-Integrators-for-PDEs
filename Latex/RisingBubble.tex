\section{Rising Bubble}

Here we will look into computing a rising bubble, in order to investigate how well these solvers work for problems in fluid dynamics.
\subsection{Formulation of Problem}
The problem is given by the following equations\cite{Bryan2002}:
\begin{align*}
    \frac{\partial u_i}{\partial t} &= -\frac{\partial u_i u_j}{\partial x_j} + u_i\frac{\partial u_j}{\partial x_j} - c_p \theta \frac{\partial \pi}{\partial x_i} \\
    \frac{\partial \pi}{\partial t} &= -\frac{\partial u_j \pi}{\partial x_j} + \pi\frac{\partial u_j}{\partial x_j} - \pi \frac{R}{c_v}\frac{\partial u_j}{\partial x_j} \\
    \frac{\partial \theta}{\partial t} &= -\frac{\partial u_j \theta}{\partial x_j} + \theta\frac{\partial u_j}{\partial x_j} - \theta (\frac{R}{c_{vml}} - \frac{R}{c_v})\frac{\partial u_j}{\partial x_j}
\end{align*}
Where Einstine summing notation is used.
We use a domain $\Omega = (0,1000) \times (0,2000)$.
We enforce reflective boundary conditions along the sides of the domain.
The initial condtions are as follows:
\begin{align*}
    u_i &= 0\\
    \pi(0,x,y) &= \pi_0(x,y)\\
    \theta &= 0
\end{align*}
where
\[
    \pi_0(x,y)= 
\begin{cases}
    1& \text{if } sqrt{(x-500)^2+(z-520)^2}\leq 50\\
    e^{ \frac{-(r-50)^2}{100^2}},              & \text{otherwise}
\end{cases}
\]
\subsection{Numerical Solutions}
We run our simulations on a finite volume mesh.
The grid is rectangular and of dimension $64\times 128$.
We use a timestep of $\tau = 1.8$

\begin{figure}[H]
    \centering
    \begin{minipage}{0.49\textwidth}
        \includegraphics[width=1\textwidth]{example-image-a} % Change filename to your image
        \caption{from first order exponential integrator with Krylov size $32$}
        \label{fig:first order 8 0.5}
    \end{minipage}\hfill
    \centering
    \begin{minipage}{0.49\textwidth}
        \includegraphics[width=1\textwidth]{example-image-b} % Change filename to your image
        \caption{from first order exponential integrator with Krylov size $64$}
        \label{fig:first order 10 0.5}
    \end{minipage}\hfill
\end{figure}